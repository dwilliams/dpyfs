% The documentation for the storage daemon for the dpyfs project
% Daniel Williams <dwilliams@port8080.net>

% Command for compiling:
% latex storaged && latex storaged && makeglossaries storaged && makeindex storaged && latex storaged && latex storaged
%                && dvipdfm storaged
% Initially I don't need the makeglossaries or makeindex, so I'll use this for now:
% latex storaged && latex storaged && dvipdfm storaged

\documentclass[letterpaper]{article}
\usepackage[letterpaper]{geometry}

\title{dpyfs \\ Storage Daemon}
\author{Daniel Williams \\ dwilliams@port8080.net}

\begin{document}
\maketitle

% Section "What?"
\section{Description}

% Section "Requirements"
\section{Requirements}

\subsection{Store Chunks}

\begin{itemize}
\item
Set chunk size - The size of each chunk stored will be of the same and will be set in the configuration file.

\item
Hash chunks for integrity checks - Each chunk will be hashed and the hashes will be checked against the request to
verify the itegrity of the chunk.

\item
File backed - All data chunks will be stored in files on the native file system.

\item
One chunk per file - Each file will store one chunk and no other data.

\item
Use hashes for names (md5-path, sha1-file) - The chunk files will be named using the hashes of the file.  The path will
use the MD5 hash and the name will use the SHA1 hash.

\item
Split path hash at three chars - The hash used for the path will be split every three chars
(i.e. \verb=???/???/???/???/???/???/???/???/???/???/??/<name>.obj= where ??? is a part of the path hash).

\item
Integrity check function for chunks - A function that can be initiated periodically will be supplied to check the
integrity of each chunk.
\end{itemize}

\subsection{RESTful Interface}

\subsubsection{Resource: /info}

\begin{itemize}
\item
Disk statistics
\begin{itemize}
\item
free space - How many bytes on the storage volume are available to the service.  This should be displayed in megabytes
(MB).

\item
total space - How many bytes the storage volume contains.  This should be displayed in megabytes (MB).

\item
free space percentage - What percent of the storage volume is available to the service.
\end{itemize}

\item
number of chunks - How many chunks are stored.  This will be updated by the create, delete, and integrity check
functions.

\item
GET only

\item
multiple formats - The information returned should be returned in multiple formats (html, json, etc.) based on the HTTP
"Accept" header.
\end{itemize}

\subsubsection{Resource: /data}

\begin{itemize}
\item
GET chunk - Allow retrieval of a chunk using the HTTP GET verb and the md5 and sha1 values of the chunk.

\item
PUT chunk (error on hash mismatch) - Allow deposit of a chunk using the HTTP PUT verb and the md5 and sha1 values of the
chunk.  If the md5 and sha1 hashes mismatch, return an error and do not store the chunk.

\item
DELETE chunk - Allow deletion of a chunk using the HTTP DELETE verb and the md5 and sha1 values of the chunk.

\item
Don't POST chunk - Currently, the HTTP POST verb will not be supported.
\end{itemize}

\subsubsection{HTTPS Support}

\begin{itemize}
\item
Certificate Authority support?  Leverage something similar to puppet?

\item
Implement this in the future.
\end{itemize}

\subsubsection{Authentication}

\begin{itemize}
\item
Implement this in the future.
\end{itemize}

% Section "Design Details"
\section{Design Details}

% Section "Integration Guide" or How I Learned to Stop Worrying and Talk to this Part
\section{Integration Guide}

% Section "Users Guide" aka how to configue this thing
\section{Users Guide}

% These sections, especially the "Guide" sections, will eventually be used for a bigger document for the system as a
% whole.

\end{document}