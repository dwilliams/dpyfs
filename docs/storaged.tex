% The documentation for the storage daemon for the dpyfs project
% Daniel Williams <dwilliams@port8080.net>

% Command for compiling:
% latex storaged && latex storaged && makeglossaries storaged && makeindex storaged && latex storaged && latex storaged
%                && dvipdfm storaged
% Initially I don't need the makeglossaries or makeindex, so I'll use this for now:
% latex storaged && latex storaged && dvipdfm storaged

\documentclass[letterpaper]{article}
\usepackage[letterpaper]{geometry}

\title{dpyfs \\ Storage Daemon}
\author{Daniel Williams \\ dwilliams@port8080.net}

\begin{document}
\maketitle

% Section "What?"
\section{Description}

% Section "Requirements"
\section{Requirements}

\subsection{Store Chunks}

\begin{itemize}
\item
Set chunk size (user defined in config)

\item
Hash chunks for integrity checks

\item
Use hashes for names (md5-path, sha1-file)

\item
File backed

\item
Integrity check function for chunks
\end{itemize}

\subsection{RESTful Interface}

\subsubsection{Resource: info}

\begin{itemize}
\item
Disk stats
\begin{itemize}
\item
free space

\item
total space

\item
inodes free

\item
inodes total
\end{itemize}

\item
number of chunks

\item
GET only
\end{itemize}

\subsubsection{Resource: data}

\begin{itemize}
\item
GET chunk

\item
PUT chunk (error on hash mismatch)

\item
DELETE chunk

\item
POST chunk (Should I do this for good measure?  I don't think I need it.)
\end{itemize}

\subsubsection{HTTPS Support}

\begin{itemize}
\item
Certificate Authority support?  Leverage something similar to puppet?

\item
Implement this in the future
\end{itemize}

\subsubsection{Authentication}

\begin{itemize}
\item
Implement this in the future
\end{itemize}

% Section "Design Details"
\section{Design Details}

% Section "Integration Guide" or How I Learned to Stop Worrying and Talk to this Part
\section{Integration Guide}

% Section "Users Guide" aka how to configue this thing
\section{Users Guide}

% These sections, especially the "Guide" sections, will eventually be used for a bigger document for the system as a
% whole.

\end{document}